\documentclass[13pt,a4paper]{article}
\usepackage[utf8]{inputenc}
\usepackage{vietnam}
\usepackage{geometry}
\geometry{letterpaper}
\title{\textbf{ĐỒ ÁN MÔN HỌC}}
\author{Luong Khanh Loc}
\begin{document}
\maketitle
\section{Cơ sở lý thuyết}
\subsection{Áp suất vỉa và vỡ vỉa}
\subsubsection{abc}

\subsection{Áp suất đáy giếng}
\subsection{Dung dịch khoan}
\subsection{Tổn thất áp suất}
\section{Các kỹ thuật khoan trong phương pháp khoan dưới cân bằng}
\subsection{Khoan bằng khí khô}
\subsection{Khoan bằng khí Ni-tơ}
\subsection{Khoan bằng khí tự nhiên}
\subsection{Khoan bằng mù}
\subsection{Khoan bằng bọt}
\subsection{Khoan bằng bọt sánh}
\subsection{Khoan bằng khí hoá lỏng}
\subsection{Flow drilling}
\subsection{Snub drilling}

\section{Các vấn đề thường gặp trong khoan dưới cân bằng}
\subsection{Hệ thống ống cuốn xoắn}
\subsection{Khoan định hướng}
\subsection{Đo địa vật lý trong khi khoan}
\subsection{Chế độ an toàn}
\subsection{Phân loại hệ thống khoan}
\subsection{Lựa chọn các kỹ thuật thích hợp}
\subsection{Kiểm soát áp suất khi khoan}

\section{Kiểm soát thông số trong quá trình khoan dưới cân bằng}
\subsection nghĩa
\end{document}
